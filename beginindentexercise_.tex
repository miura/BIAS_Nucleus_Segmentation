\begin{indentexercise}
{3}

\begin{enumerate}
\item Modify the code above so that the function calculates m to the power of n. Use command pow(m, n) for this. Run the code.

\item Change the name of function to calc2 and run the code. If there is error, fix the code. 

\end{enumerate}

\end{indentexercise}

I added three lines after the precious code. Line 6 declares that I am trying to define a new user-defined function with the name of command "calc1" with two required arguments n and m. Using those variables in the argument, calculation is done and then the result is \textbf{return} ed. Functions normally has input and output. In this case, inputs are n and m, output is the calculation result.

In the same way, we could convert the segmentation macro and create a function that takes an imageID as input, do processing to do segmentation, and then returns an image ID as output. Here is the code.